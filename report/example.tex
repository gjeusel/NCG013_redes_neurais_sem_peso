\documentclass[a4paper, 11pt]{article}

\usepackage{fullpage} % Package to use full page
\usepackage{parskip} % Package to tweak paragraph skipping
\usepackage{tikz} % Package for drawing
\usepackage{amsmath}
\usepackage{hyperref} % for hyperlink and refs
\hypersetup{
    colorlinks = true,
    linkcolor = brown,
    linkbordercolor = {white},
}

\usepackage{multicol} % mutlicol management

% Brazilian encoding :
\usepackage[brazilian]{babel} % English language/hyphenation
\usepackage[T1]{fontenc} % Use 8-bit encoding that has 256 glyphs
\usepackage[utf8]{inputenc}

% Symbols :
\usepackage{gensymb}

%%%%%%%%% Pictures management : %%%%%%%%%
\usepackage{graphicx} % to insert png
\usepackage{adjustbox} % for better figure positionning
\usepackage[a4paper]{geometry} % for geometry changes in only one page (\newgeometry{...})
\usepackage[space]{grffile} % allow space in figure path
\usepackage{float}% Prevent pictures to moves to next page automatically

%%%%%%%%% Algorithm format : %%%%%%%%%
\usepackage{algpseudocode}


%%%%%%%%% Cosmetic Headers and footers : %%%%%%%%%
\usepackage{fancyhdr} % Custom headers and footers
\pagestyle{fancy} % Makes all pages in the document conform to the custom headers and footers
\fancyhead[R]{}
\fancyhead[L]{}
\fancyfoot[L]{} % Empty left footer
\fancyfoot[C]{} % Empty center footer
\fancyfoot[R]{\thepage} % Page numbering for right footer
\renewcommand{\headrulewidth}{0pt} % Remove header underlines
\renewcommand{\footrulewidth}{0pt} % Remove footer underlines
\setlength{\headheight}{13.6pt} % Customize the height of the header
\setlength{\footskip}{20pt} % Customize the skip of the footer
\setlength{\textheight}{650pt} % Customize the height of the text

\usepackage{titling} % Title package

%%%%%%%%% Personalized commands : %%%%%%%%%
\newcommand{\horrule}[1]{\rule{\linewidth}{#1}} % Create horizontal rule command with 1 argument of height

%--------------------------------------------------
%    TITLE SECTION
%--------------------------------------------------
\pretitle{
  \begin{figure}[H] % floating picture
  \begin{center}
  \includegraphics[width=0.60\textwidth]{figures/UFRJ-noticias.png}  
  \includegraphics[width=0.70\textwidth]{figures/pipe_scheme_states.jpg}  
\\[\bigskipamount]
}
\posttitle{\end{center} \end{figure}}

\title{
\normalfont \normalsize
%\textsc{Universidade Federal do Rio de Janeiro} \\ [25pt] % Your university, school and/or department name(s)
\horrule{0.5pt} \\[0.4cm] % Thin top horizontal rule
\huge \textbf{Escoamentos Bifásicos}\\ % The assignment title
\horrule{2pt} \\[0.5cm] % Thick bottom horizontal rule
}

\author{Guillaume Jeusel \\ \\
Palestra : TransCalor II \\ \\
Professor : Antonio Mac Dowell De Figueiredo}

\date{\normalsize\today} % Today's date or a custom date


\begin{document}

\maketitle % Print the title
\newpage

%------------------------------------------------------------------------------
\section{Estudo dos parâmetros no caso de Fases separadas - Película e Núcleo}
O objetivo dessa seção de determinar a influencia das principais variáveis do modelo de escoamento bifásico anular entre si.
Ou seja, a influencia :
\begin{itemize}
  \item do numero de Reynolds no Núcleo
  \item do numero de Reynolds na Película
  \item do gradiente de pressão
  \item da espessura da película
\end{itemize}
Nos vamos nos concentrar somente nos parâmetros adimensionais para ter uma interpretação geral.
Alem disso, nos vamos considerar que a espessura da película $\delta$ é sempre dada. Isso é porque a determinação do $\delta$ sendo dados dois outras variáveis é muito difícil e envolve sempre o obtenção das raízes de um polinômio de grau 5 de dois variáveis (cf equação 1.33).

As hipoteses físicas desse modelo são as seguintes :
\begin{enumerate}
  \item Escoamento em regime permanente
  \item Escoamento completamente desenvolvido
  \item Escoamento laminar
  \item Fluido newtoniano
  \item Fluido incompressível
\end{enumerate}

%------------------------------------------------------------------------------
\subsection{Faixas de valores fisicamente aceitáveis}
O primeiro trabalho que tem que ser feito é de encontrar as faixas de valores fisicamente aceitáveis para esses parâmetros.
Para fazer isso, nos vamos nos colocar em casos particulares e tentar dar uma interpretação física ao que acontece.

Seja um escoamento bifásico anular num duto reto vertical de raio $R=1cm$ com uma película de aguá liquida e um núcleo de vapor de aguá.

\begin{center}
\begin{tabular}{lccc}
	\hline
    variável & caract. & valor & unidade \\
    \hline
    $\theta$ & angulo do tubo & 90 & $\degree$ \\
    rho\_p & massa volúmica na película & 958 & $kg/m^3$ \\
    rho\_n & massa volúmica no núcleo & 0.59 & $kg/m^3$ \\
    mu\_p & viscosidade na película & $2.83*10^{-4}$ & $Pa.s$ \\
    mu\_n & viscosidade no núcleo & $1.2*10^{-5}$ & $Pa.s$ \\
    \hline
\end{tabular}
\end{center}

\bigskip
\textbf{Espessuras consideradas} :\\
$\eta=[0.1, 0.075, 0.05, 0.025, 0.01, 0.0075, 0.0025, 0.001]$ com $\eta = \frac{\delta}{R}$

\bigskip
\textbf{Formulações} usadas para a determinação dos \textbf{números de Reynolds} :
\begin{itemize}
\item Equaçao 1.33 para o Numero de Reynolds no \textbf{Núcleo} :
\begin{figure}[H]
  \begin{center}
  \includegraphics[width=\textwidth]{figures/eq_1_33.jpg}  
\end{center} \end{figure}
\item Equação 1.34 para o Numero de Reynolds na \textbf{Película}
\begin{figure}[H]
  \begin{center}
  \includegraphics[width=\textwidth]{figures/eq_1_34.jpg}  
\end{center} \end{figure}
\end{itemize}

%------------------------------------------------------------------------------
\subsubsection{Sem efeito viscoso na interface}
Nesse caso, o gradiente de pressão adimensionalisado é obtido pela equação 1.30 sendo a tensão cisalhante nula na interface :
\begin{figure}[H]
  \begin{center}
  \includegraphics[width=0.5\textwidth]{figures/eq_1_30.jpg}  
\end{center} \end{figure}
E ele acarreta ser constante igual a $-\rho^*sen \theta$.

Resultados obtidos :
\begin{figure}[H]
  \includegraphics[width=\textwidth]{figures/b_caso_1.png}  
\end{figure}

Um Reynolds positivo pode ser interpretado por um escoamento ascendente, e um Reynolds negativo pode ser interpretado por um escoamento descendente.
Por consequência, no caso de ausência de tensão cisalhante na interface, ocorre que o escoamento no Núcleo é descendente, e ascendente na película.
Essa configuração se justifica porque se não tem tensão cisalhante entre esses fluidos, e que o escoamento é laminar, então eles não interagem entre si dinamicamente falando.

%------------------------------------------------------------------------------
\subsubsection{Sem fluxo mássico no núcleo}
Nesse caso, o gradiente de pressão adimensionalisado é obtido pela equação 1.33 sendo o Reynolds no núcleo nulo.

Resultados obtidos :
\begin{figure}[H]
  \includegraphics[width=\textwidth]{figures/b_caso_2.png}  
\end{figure}

É interessante apontar que quando tem um fluxo mássico nulo no núcleo, o escoamento na película é ascendente ($Re_p$ positivo).

%------------------------------------------------------------------------------
\subsubsection{Sem fluxo mássico na película}
Nesse caso, o gradiente de pressão adimensionalisado é obtido pela equação 1.34 sendo o Reynolds na película nulo.

Resultados obtidos :
\begin{figure}[H]
  \includegraphics[width=\textwidth]{figures/b_caso_3.png}  
\end{figure}

Deve-se apontar que quando o fluxo mássico na película é nula, o escoamento no núcleo é descendente.

%------------------------------------------------------------------------------
\subsubsection{Velocidade nula na interface}
Nesse caso, o gradiente de pressão adimensionalisado é obtido pela equação 1.32 sendo a velocidade adimensional na interface nula :

\begin{figure}[H]
  \begin{center}
  \includegraphics[width=0.8\textwidth]{figures/eq_1_32.jpg}  
\end{center} \end{figure}

Resultados obtidos :
\begin{figure}[H]
  \includegraphics[width=\textwidth]{figures/b_caso_4.png}  
\end{figure}

%------------------------------------------------------------------------------
\subsubsection{Velocidade nula na interface}
No caso de uma parede considerada perfeitamente lisa, o gradiente de pressão adimensionalisado é obtido pela equação 1.29 sendo a tensão cisalhante na parede nula :

\begin{figure}[H]
  \begin{center}
  \includegraphics[width=0.6\textwidth]{figures/eq_1_29.jpg}  
\end{center} \end{figure}

Resultados obtidos :
\begin{figure}[H]
  \includegraphics[width=\textwidth]{figures/b_caso_5.png}  
\end{figure}

%------------------------------------------------------------------------------
\subsubsection{Conclusões}
Todas essas curvas são obtidas executando o script python que se chama \textbf{questao\_1\_b.py} e escolhendo o caso :
\begin{figure}[H]
  \includegraphics[width=\textwidth]{figures/screen_script_b.jpg}  
\end{figure}

É interessante de notar que os parâmetros são muito sensíveis a qualquer pequena variação. Por consequência, a determinação de faixas para obter resultados fisicamente aceitaveis acarreta ser bem complicado.
No entanto, uma tentativa para obter eles é fornecida no script python \textbf{questao\_1\_a.py}.

\newpage
%------------------------------------------------------------------------------
\section{Estudo do modelo de Lockhart \& Martinelli}

\subsection{Introdução}
A correlação de Lockhart \& Martinelli foi desenvolvida para escoamentos isotérmicos de misturas bifásicas líquido-gás, em regime permanente.

A obtenção de uma expressão para o multiplicador bifásico do modelo de Lockhart \& Martinelli se faz a partir das expressões das perdas de carga do líquido e do gás.
O raciocínio para obter o $ X^2:= \frac{tensao cisalhante da Pelicula na parede}{tensao cisalhante do Nucleo na parede} := \frac{\tau_{1,P}}{\tau_{1,N}} $ pode ser encontrada na literatura.
Chegamos na expressão seguinte :
$ X^2 = \frac{f_p*Re_p^2}{f_n*Re_n^2}*\frac{\rho^*}{\mu^{*^2}} $

E ai vem a questão do calculo dos coeficientes de atrito $f_p$ e $f_n$.
Generalmente, esses valores são dado pela fórmula de \textbf{Blasius} da forma
$f = Re^{-n}$.
No entanto, uma formulação mais elaborada da lei de atrito de \textbf{Darcy-Weisbach} pode ser usada para ter resultados mais justos :
\begin{center}
\[f = ( -2*log_{10}(\frac{k}{3.7*D} - \frac{5.16}{R_e}*log_{10}(\frac{k}{3.7*D} + \frac{5.09}{R_e^{0.87}})) )^{-2}\]
\end{center}
com $k$ a rugosidade especifica (foi pegada aquela do aço enferrujado $k=5*10^{-4}$).
Deve ser considerado que essa correlação só vale para Reynolds maiores do que 3000.

%------------------------------------------------------------------------------
\subsection{Resultados}

O script python \textbf{questao\_2.py} permite obter o grafico seguinte :
\begin{figure}[H]
  \begin{center}
  \includegraphics[width=\textwidth]{figures/screen_script_2.jpg}  
\end{center} \end{figure}
\begin{figure}[H]
  \begin{center}
  \includegraphics[width=\textwidth]{figures/khi_2_Re_n_dado.png}  
\end{center} \end{figure}

Aparentemente, a lei de Blasius sub-estima o valor do $X^2$.


\newpage
%------------------------------------------------------------------------------
\section{Transferência de calor no Escoamento Bifásico com Mudança de fase}

O objetivo desse trabalho é a determinação dos coeficientes de transferência de calor usando varias correlações sendo o regime bifásico escoando.
Nos vamos nos colocar no caso de \textbf{escoamento em ebulição saturado} (ou seja 'Saturated Flow Boiling', com um duto vertical.

\begin{figure}[H]
  \begin{center}
  \includegraphics[width=0.08\textwidth]{figures/saturated_flow_boiling.jpg}  
\end{center} \end{figure}

O escoamento do refrigerente R-22 com fluxo massica de $G=200 kg/m^2.s$ foi pegado, cujas caracteristicas na pressao $P=2*10^5 Pa$ sao as seguintes :

\begin{center}
\begin{tabular}{lccc}
	\hline
    variavel & caract. & valor & unidade \\
    \hline
    T\_sat & temperatura de saturação & 260 & K \\
    P\_sat & pressão de saturação & $2.0965*10^5$ & Pa \\
    rho\_l & massa volúmica da fase liquida & 1360 & $kg/m^3$ \\
    rho\_v & massa volúmica da vapor & 9.59 & $kg/m^3$ \\
    mu\_l & viscosidade do liquido & $282*10^{-6}$ & $N.s/m^2$ \\
    mu\_v & viscosidade da vapor & $10.9*10^{-6}$ & $N.s/m^2$ \\
    h\_lv & entalpia especifica de vaporização & $226 * 10^3$ & J/kg \\
    c\_pl & capacidade calorifica do liquido & $1.13*10^3$ & J/kg.K \\
    k\_l & condutividade térmica do liquido & 0.109 & W/m.K \\
    sigma & tensão de superfície & 0.0155 & N/m \\
    \hline
\end{tabular}
\end{center}

Fora implementado as correlações seguinte :
\begin{enumerate}
  \item Chen
  \item Shah
  \item Schrock and Grossman
  \item Gungor and Winterton
  \item Bjorge, Hall and Rohsenow (mais nao dou resultados bons)
  \item Kandlikar
\end{enumerate}

Executando o script python \textbf{questao\_3.py} :
\begin{figure}[H]
  \begin{center}
  \includegraphics[width=\textwidth]{figures/screen_script_3.jpg}  
\end{center} \end{figure}
Foi obtidos as seguintes curvas :

\begin{figure}[H]
  \begin{center}
  \includegraphics[width=\textwidth]{figures/h_correlations_result.png}  
\end{center} \end{figure}

Pode-se ver que as correlações de Chen e Shah são muito próximos, como aqueles do Schrock e Grossman. No entanto, a correlação de Kandlikar não parece dar resultados bons.

\end{document}

